\documentclass[a4paper]{article}

\addtolength{\oddsidemargin}{-0.7in}
\addtolength{\evensidemargin}{-0.7in}
\addtolength{\textwidth}{1.75in}
\addtolength{\topmargin}{0in}
\addtolength{\textheight}{1.75in}
\setlength{\parindent}{0pt}

\begin{document}

\subsection*{Observation about logistic regression stability}

\begin{itemize}
\item Logistic regression doesn't work well when features vary in magnitude.
\item Our power-based classifier has $\textrm{AUC}=0.7$ for R1111M FR1, LOSO.
\item If we double the number of features by adding randomly generated noise from ${\cal N}(0,1)$, AUC drops to
about $0.68$.
\item Now, if we double the magnitude of noise (i.e., use ${\cal N}(0,2)$ instead of ${\cal N}(0,1)$ for the noise generation)
AUC drops to $0.63$.
\item If you multiply the noise magnitude by $5$, AUC drops to $0.57$.
\end{itemize}

Conclusion: it's important to normalize PPC features in some way as they differ a lot for different frequencies and
pairs of bipolar pairs.

\end{document}
